\section{Predictive Markets}
\label{sec:predictive_markets}

A prediction market, as commonly defined and the focus of this work, is a market where participants can trade and exchange contracts, and the payoffs are contingent on the outcomes of future events(\citeauthor{wolfers2004prediction}). This type of market has caught the attention of the media and various technology companies in recent years due to its ability to make more accurate predictions through its pricing. According to an article from \textcite{Roose_2023}, "Prediction markets offer a better way to search for truth — rewarding those who are good at forecasting by allowing them to make money off those who are bad at it, while settling on the facts in an unbiased way".

We could also understand prediction markets as a type of bet on future events. However, the significant difference from a traditional bet lies in how prices are set. A clear way to understand how prices are set is through the model proposed by  /textcite{wolfers2004prediction}  of a simple prediction market, where traders buy and sell an all-or-nothing contract that pays  /1 if the future event predicted by the agent occurs and nothing otherwise.

Under this model, traders have logarithmic utility and endogenously derive their trading activity given the contract price is \( \pi \). Thus, when deciding how many contracts \( x \) to buy, traders solve the following problem:
\begin{equation}
    \max U = q_j \log(y_j + x(1 - \pi)) + (1 - q_j) \log(y_j - x\pi)
\end{equation}
Where:

- Trader \( j \)
- Belief of trader \( j \) that the event will occur \( q_j \)
- Wealth levels (\( y \))

An optimum is found where:
\begin{equation}
    x_j^* = \frac{y_j(q_j - \pi)}{\pi(1 - \pi)}
\end{equation}

The prediction market is in equilibrium when supply equals demand:
\begin{equation}
    \int_{-\infty}^{\infty}\frac{y_j(q_j - \pi)}{\pi(1 - \pi)}f(q)dq = \int_{-\infty}^{\infty}\frac{y_j(\pi -q_j)}{\pi(1 - \pi)}f(q)dq\infty}^{\infty} q f(q) dq = q^-
\end{equation}

And the next equilibrium price is reached:

\begin{equation}
    \pi = \int_{-\infty}^{\infty} q f(q) dq = q^-
\end{equation}

The previous model not only allows us to observe the process by which the price is established in a predictive market but also reaches a very relevant conclusion in which market prices are equal to the average belief among traders.

\lipsum[2]

\subsection{Winner-Takes-All Contracts}
\label{subsec:winner_takes_all_contracts}

\lipsum[2]
