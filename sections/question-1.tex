\section{Dynamics of a Demand Shock}

Consider the market for a single good. Market demand is \( X(p) = 1500 - 50p \), where \( p \) is the price of the good. There are many potential firms, each with cost function \( c(q_k) = 100 + q_k^2/4 \).

%%%%%%%%%%%%%%%%%%%%%%%%%%%%%%%%%%%%%%%%%%%%%%%%%%%%%%%%%%%%%%%%%
\begin{tcolorbox}
    (a) Calculate the long-run equilibrium, in which firms are free to enter and exit. How much does each firm produce (\( q^* \))? What is the equilibrium price (\( p^* \))? How many firms enter (\( K^* \))?
\end{tcolorbox}

Each firm \( k \) will produce in their efficient point. This means:

\begin{eqnarray*}
MCG_k(q_k) = AC_k(q_k)
\end{eqnarray*}

From our cost function \( c_k(q_k) \):
\begin{eqnarray*}
MCG_k(q_k) &=& \frac{d}{dq_k} c_k(q_k) = \frac{q_k}{2}\\
AC_k(q_k) &=& \frac{c_k(q_k)}{q_k} = \frac{100}{q_k} + \frac{q_k}{4}\\
\end{eqnarray*}

Solving for the firm \( k \):

\begin{eqnarray*}
MCG_k(q_k) &=& AC_k(q_k)\\
\frac{q_k}{2} &=& \frac{100}{q_k} + \frac{q_k}{4}\\
\frac{q_k^2}{4} &=& 100\\
q_k &=& 20
\end{eqnarray*}

\begin{myanswerbox}
    So each firm \( k \) is producing \( q^* = 20 \).
\end{myanswerbox}

We know that the supply function for the firm \( k \) is \( p = MCG_k(q_k) \). The price \( p \) is:

\begin{eqnarray*}
p &=& MCG_k(q_k) = MCG_k(20)\\
p &=& \frac{q_k}{2} = \frac{20}{2} = 10
\end{eqnarray*}

\begin{myanswerbox}
    The equilibrium price is \( p^* = 10 \).
\end{myanswerbox}

At this price, the demand is:

\begin{eqnarray*}
X(p) &=& 1500 - 50p\\
X(10) &=& 1500 - 50(10)\\
X(10) &=& 1000
\end{eqnarray*}

\begin{myanswerbox}
  So the demand is \( X(10) = 1000 \).
\end{myanswerbox}


%%%%%%%%%%%%%%%%%%%%%%%%%%%%%%%%%%%%%%%%%%%%%%%%%%%%%%%%%%%%%%%%%
%%%%%%%%%%%%%%%%%%%%%%%%%%%%%%%%%%%%%%%%%%%%%%%%%%%%%%%%%%%%%%%%%
\begin{tcolorbox}
    (b) Suppose demand rises. New demand is given by \( \hat{X}(p) = 1800 - 50p \). Suppose demand rises. New demand is given by \( \hat{X}(p) = 1800 - 50p \).\\
    
    In the very-short run, each firms' output is fixed. What is equilibrium price(\( p_v \))? How much profit does each firm make (\( \pi_v \))?
\end{tcolorbox}

In the very short-run there are only \( K = 50 \) producing \( q_k = 20 \) goods each, so the aggregate supply is \( Q = 1000 \).
At that level of aggregated supply, and given the new aggregated demand:

\begin{eqnarray*}
\hat{X}(p) = 1800 - 50p = 1000 
\end{eqnarray*}

\begin{myanswerbox}
    The new equilibrium price is \( p_v = 16 \)
\end{myanswerbox}

At that price, the profit of each firm is:

\begin{eqnarray*}
\pi_v = 16(20) - \left(100 + \frac{20^2}{4}\right)
\end{eqnarray*}

\begin{myanswerbox}
So each firm is making a profit of \( \pi_k = 120 \). This level of profit is not sustainable because, as time passes, firms have incentives to produce one extra unit and take advantage of that price.
\end{myanswerbox}


%%%%%%%%%%%%%%%%%%%%%%%%%%%%%%%%%%%%%%%%%%%%%%%%%%%%%%%%%%%%%%%%%
%%%%%%%%%%%%%%%%%%%%%%%%%%%%%%%%%%%%%%%%%%%%%%%%%%%%%%%%%%%%%%%%%
\begin{tcolorbox}
    (c) In the short run, each firm can adjust its output to maximize its profits, but there is no entry or exit. What is equilibrium price (\( p_s \))? How much does each firm produce (\( q_s \))? How much profit does each firm make (\( \pi_s \))?
\end{tcolorbox}

The number of producers remains the same, so \( K = 50 \). Now each firm wants to maximize their profit.

\begin{eqnarray*}
\pi_k(q_k) = pq_k - c_k(q_k)
\end{eqnarray*}

In order to maximize:

\begin{eqnarray*}
\frac{d}{dq_k} \pi_k(q_k) &=& 0 \implies p - \frac{q_k}{2} = 0\\
q_k &=& 2p
\end{eqnarray*}

So the optimal aggregated supply is:

\begin{eqnarray*}
Q(p) = 50(2p) = 100p
\end{eqnarray*}

In equilibrium \( \hat{X}(p) = Q(p) \):

\begin{eqnarray*}
1800 - 50p = 100p  \implies p_s = 12
\end{eqnarray*}

At this price, the aggregated demand is:

\begin{eqnarray*}
\hat{X}(p_s) = 1800 - 50p_s = 1800 - 50(12) = 1200
\end{eqnarray*}

So each firm produces:

\begin{eqnarray*}
q_s = \frac{1200}{50} = 24
\end{eqnarray*}

And makes a profit of:

\begin{eqnarray*}
\pi_s = (12)(24) - \left(100 - \frac{24^2}{4}\right) = 44
\end{eqnarray*}

\begin{myanswerbox}
    As a summary, in the short-run, the new equilibrium is:
    \begin{eqnarray*}
    K &=& 50\\
    q_s &=& 24\\
    p_s &=& 12\\
    \pi_s &=& 44
    \end{eqnarray*}
\end{myanswerbox}
%%%%%%%%%%%%%%%%%%%%%%%%%%%%%%%%%%%%%%%%%%%%%%%%%%%%%%%%%%%%%%%%%
%%%%%%%%%%%%%%%%%%%%%%%%%%%%%%%%%%%%%%%%%%%%%%%%%%%%%%%%%%%%%%%%%
\begin{tcolorbox}
   (d) In the long run, there is free entry and exit. How much does each firm produce (\( q_e \))? What is equilibrium price (\( p_e \))? How many firms enter (\( K_e \))?
\end{tcolorbox}
    
In the long run, assuming that each firm has the same cost function, each firm will produce \( q_c = 20 \) at a price \( p_c = 10 \), which is the efficient point identified prior to the expansion of demand.\\
    
To find the quantity of firms producing:
\begin{eqnarray*}
K_c = \frac{\hat{X}(p_c)}{q_c} = \frac{1800 - 50(10)}{20} = 65
\end{eqnarray*}

\begin{myanswerbox}
    In the long-run there will be \( K_c = 65 \) firms producing.
\end{myanswerbox}

%%%%%%%%%%%%%%%%%%%%%%%%%%%%%%%%%%%%%%%%%%%%%%%%%%%%%%%%%%%%%%%%%