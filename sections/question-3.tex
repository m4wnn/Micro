\section{Double Marginalization}

A manufacturer M sells his product to consumers through a retailer R. First, the manufacturer sets a wholesale price $w \geq 0$. The retailer sees this and chooses the final price p $\geq$ w. The demand function is $Q(p) = 1 - p$. For simplicity, assume the manufacturer and retailer have zero cost. The profit functions are thus given by $\pi M = w(1 - p)$ for the manufacturer and $\pi R = (p - w)(1 - p)$ for the retailer.

\begin{tcolorbox}
    (a) Given w what is the optimal price p of the retailer?
\end{tcolorbox}

\begin{eqnarray*}
\pi_R &=& (p - w)(1 - p)\\
\pi_R &=& p-p^2-w+wp\\
\frac{d\pi_R}{dp} &=& 1-2p+w = 0\\
\frac{1+w}{2} &=& p\\
\end{eqnarray*}

\begin{myanswerbox}
    The optimal price of the retailer is $\frac{1+w}{2}$.
\end{myanswerbox}

\begin{tcolorbox}
    (b) Using backwards induction, what is the optimal wholesale price w of the manufacturer?
\end{tcolorbox}

\begin{eqnarray*}
\pi_M &=& w(1 - p)\\
\frac{1+w}{2} &=& p\\
\pi_M &=& w(1 - \frac{1+w}{2})\\
\pi_M &=& (w - \frac{w+w^2}{2})\\
\pi_M &=& w - \frac{w}{2}-\frac{w^2}{2}\\
\frac{d\pi_M}{dw} &=&  1 - \frac{1}{2}-\frac{2w}{2}=0\\
w&=&\frac{1}{2}=0.5\\
p&=&0.75
\end{eqnarray*}

\begin{myanswerbox}
    The optimal wholesale price of the manufacturer is 0.75.
\end{myanswerbox}

\begin{tcolorbox}
    (c) What are the firms' profits in equilibrium?
\end{tcolorbox}

\begin{eqnarray*}
\pi_R &=& (p - w)(1 - p)\\
\pi_R &=& (0.75 - 0.5)(1 - 0.75)\\
\pi_R &=& (0.25)(0.25)\\
\pi_R &=& 0.0625\\
\\
\pi_M &=& w(1 - p)\\
\pi_M &=& 0.5(1 - 0.75)\\
\pi_M &=& 0.5(0.25)\\
\pi_M &=& 0.125
\end{eqnarray*}

\begin{myanswerbox}
    The firms' profits in equilibrium are $\pi_R = 0.0625$ and $\pi_M = 0.125$.
\end{myanswerbox}

\begin{tcolorbox}
    (d) Now assume that manufacturer and retailer integrate vertically and charge a price p to maximize joint profits $\pi(p) = p(1 - p)$. What is the optimal price p?
\end{tcolorbox}

\begin{eqnarray*}
\pi(p) &=& p(1 - p)\\
\pi(p) &=& p - p^2\\
\frac{d\pi(p)}{dp} &=& 1 - 2p=0\\
p &=& \frac{1}{2}\\
\pi(p) &=& 0.25\\
\end{eqnarray*}

\begin{myanswerbox}
    The optimal price is 0.5 and the joint profit is 0.25.
\end{myanswerbox}

\begin{tcolorbox}
    (e) How do industry profits in the vertically integrated firm compare to the equilibrium in (c)? Explain the difference.
\end{tcolorbox}

\begin{eqnarray*}
\pi_R + \pi_M= 0.0625 + 0.125= 0.1875\\
\pi(p) = 0.25\\
\end{eqnarray*}

\begin{myanswerbox}
    The increase in total profit due to vertical integration (from  0.1875 to 0.25) exemplify the economic theory that suggests vertical integration can yield advantages by mitigating the inefficiencies linked to double marginalization, thereby improving the integrated entity's profitability.
\end{myanswerbox}