\section{AirBnB Regulation}

\begin{tcolorbox}
    One can think of AirBnB as a market. What are the key aspects of the goods being sold? What
    are the types of buyers and sellers? What are their alternatives if they aren’t on AirBnB?\\

    You are the mayor of Santa Monica. You are worried that AirBnB is pushing up prices for long-term
    rentals for residents, changing the character of the city and causing a nuisance for residents. On the
    other hand, tourists are important for the city and you AirBnB is useful in more efficiently using real
    estate. What are some of the possible dimensions for regulations? What would you recommend?
\end{tcolorbox}

The Airbnb market is driven by the interaction of two factors: demand, which is made up of people looking for short-term housing in a particular city, and supply, which is homeowners offering short-term lodging services. On the supply side, there are those who want to make money by renting out their properties for brief periods of time, either because they are underutilized or have extra space. On the other hand, short-term visitors, business travelers, and tourists are usually the ones in demand for reasonably priced and adaptable lodging in the city.\\

In the Airbnb industry, credibility is built by a thorough review system that assesses variables including price, hospitality, and accessibility. In order to protect the interests of hosts and renters, the platform also includes security measures that handle things like payments, damage policies, and fraud protection.\\

Therefore, Airbnb functions as a trusted marketplace where hosts and renters can easily engage and obtain the desired services. It exists as an As a result, Airbnb serves as a reliable platform where hosts and renters may interact and get the services they need. It is available as a substitute for other lodging choices, which range in cost and amenities, including hotels (which are frequently more expensive), rental agencies, hostels, and owner-rented vacation homes. For lodge owners looking to make money from underused properties, especially in pricey cities, Airbnb is a useful resource.\\

But the abundance of Airbnb listings in big cities has sparked worries, especially about rising real estate costs and dwindling supply of cheap accommodation. As a result, cities like Los Angeles, Amsterdam, or Paris have put in place a variety of regulations to strike a balance between the advantages of the market and the requirement to safeguard homeowners in the property market, which is a significant and relatively inelastic market.\\

One notable regulation involves limiting the number of available days for renting, aiming to curb the shift from long-term leasing to short-term leasing. This restriction addresses the tendency of property owners to favor short-term rentals for higher daily rates, potentially driving up long-term rental prices. By imposing limits, property owners have fewer incentives to transition from long-term to short-term rentals.\\

Another regulatory approach, as observed in Amsterdam, involves zoning restrictions. For instance, in Santa Monica, regulations could be implemented to restrict short-term rentals in residential areas, preserving residential prices without hindering the influx of tourists. This approach mirrors successful strategies employed in areas like The Red Zone in Amsterdam.\\

Furthermore, putting in place a thorough registration system would make it possible to keep an eye on the market for short-term rentals and enable efficient taxes. All of these actions are intended to control the Airbnb industry without completely eradicating it. Achieving a balance between providing housing options to meet demand and protecting locals from rising rental and property costs is the aim. Since Airbnb is frequently used in place of hotels and hostels, these rules provide guests flexibility in their lodging options while protecting locals in the approved areas, like Santa Monica.\\
