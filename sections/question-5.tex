\section{Ransoms}

\begin{tcolorbox}
    Under Italian law, families are barred from paying ransom to kidnappers. Why does this law exist? Is it a good law?
\end{tcolorbox}


The Italian law referred to is known as the "Anti-Kidnapping Law." It was implemented in response to a rising wave of kidnappings, particularly noticeable during the 1970s and 1980s, when Italy faced serious kidnapping problems perpetrated by both common criminals and terrorist groups like the Red Brigades. The main aim of this law was to minimize incentives for committing kidnappings by eliminating the possibility of families paying ransoms. Additionally, since the law also demands cooperation with authorities, it presumably increases the risk of capture for kidnappers.

In a context without this law, families faced a situation where the cost of not cooperating with the kidnappers was the life of their relative, while the cost of cooperating was paying a ransom. Kidnappers were aware of this dilemma, so to coerce the families' decision, they made credible threats, thereby creating a balance skewed in favor of cooperating with the kidnappers.

With the Italian law, the government aims to increase the cost of cooperating with kidnappers to a point where the threat's credibility might be insufficient for families to decide to pay the ransom. In this scenario, the outcome depends on the family's fear of the law and the credibility and brutality of the kidnappers.

In cases where kidnappers are not credible, the law might be effective as families would incur a high cost if they collaborate rather than reporting to authorities. Such cases could involve kidnappings of middle or upper-class families, where the kidnapping is random, and its effectiveness depends on the speed of ransom payment.

In scenarios where kidnappers are credible, the law might not be effective, as families are aware of the consequences of non-cooperation. This could be the case for families with some kind of relationship with the kidnappers, where the purpose of the kidnapping is not solely to extract money but also to obtain information or enact vengeance.

Another condition necessary for the effectiveness of this law is the authorities' ability to recover kidnapped relatives unharmed. If their capability is low, the cost of not cooperating with kidnappers remains high, rendering the law ineffective.