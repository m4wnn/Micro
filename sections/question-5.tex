\section{AirBnB Regulation}

\begin{tcolorbox}
    One can think of AirBnB as a market. What are the key aspects of the goods being sold? What
    are the types of buyers and sellers? What are their alternatives if they aren’t on AirBnB?\\

    You are the mayor of Santa Monica. You are worried that AirBnB is pushing up prices for long-term
    rentals for residents, changing the character of the city and causing a nuisance for residents. On the
    other hand, tourists are important for the city and you AirBnB is useful in more efficiently using real
    estate. What are some of the possible dimensions for regulations? What would you recommend?
\end{tcolorbox}

The Airbnb market operates through the interaction between supply, represented by homeowners offering short-term lodging services, and demand, comprised of individuals seeking temporary accommodations in a specific city. The supply side involves people aiming to generate income by renting their property for short durations, either because they are not utilizing the space or have additional room to spare. Conversely, the demand typically consists of tourists, business travelers, or short-term visitors seeking affordable and flexible accommodation in the city.\\

Credibility in the Airbnb market is established through a robust system of reviews, evaluating factors such as accessibility, hospitality, and pricing. Additionally, the platform incorporates security features to safeguard the interests of both hosts and renters, addressing aspects like payments, damage policies, and protection against scams.\\

Therefore, Airbnb functions as a trusted marketplace where hosts and renters can easily engage and obtain the desired services. It exists as an alternative to other accommodation options such as hotels (often with higher prices), rental agencies, hostels, or vacation rentals by owners, each varying in terms of pricing and facilities offered. Airbnb serves as a valuable market for lodge owners seeking revenue from underutilized properties, particularly in expensive cities.\\

However, the proliferation of Airbnb listings in major cities has raised concerns, particularly regarding increased housing prices and reduced availability of affordable housing. Consequently, cities like Los Angeles, Amsterdam, or Paris have implemented various restrictions to balance the benefits of the market with the need to protect residents within the housing market, which is a significant and relatively inelastic market.\\

One notable regulation involves limiting the number of available days for renting, aiming to curb the shift from long-term leasing to short-term leasing. This restriction addresses the tendency of property owners to favor short-term rentals for higher daily rates, potentially driving up long-term rental prices. By imposing limits, property owners have fewer incentives to transition from long-term to short-term rentals.\\

Another regulatory approach, as observed in Amsterdam, involves zoning restrictions. For instance, in Santa Monica, regulations could be implemented to restrict short-term rentals in residential areas, preserving residential prices without hindering the influx of tourists. This approach mirrors successful strategies employed in areas like The Red Zone in Amsterdam.\\

Additionally, implementing a comprehensive registration system would allow monitoring of short-term rentals in the market, facilitating effective taxation. These measures collectively aim to regulate the Airbnb market without eliminating its presence. The goal is to strike a balance, offering accommodation options for demand while safeguarding residents from escalating rental and housing prices. Given that Airbnb is widely accepted as a substitute for hotels and hostels, such regulations allow for flexibility in choices for visitors while ensuring the protection of local residents in the designated zones, such as Santa Monica.\\
