\section{Revising the Assumptions About Predictive Markets}
\label{sec:revising_assumptions}

\subsection{Efficency and Arbitrage opportunities}
\label{subsec:efficency_and_arbitrage_opportunities}

Predictive markets, such as Polymarket, should theoretically work well and be very efficient, so arbitrage opportunities would not be expected to be persistent(\citeauthor{luckner2008arbitrage}). These are designed to be extremely efficient because they combine data from a wide range of participants, which results in the formation of prices that accurately reflect the likelihood of future events. This implies that market participants seeking to capitalize on these differences should quickly eliminate any price discrepancies that could create arbitrage opportunities.

However, it is essential to understand that no market can operate perfectly all the time. In practice, brief arbitrage opportunities may arise because information updates are delayed, market liquidity constraints are imposed, or market participants' responses to new information are delayed. Fast and agile arbitrageurs can exploit these opportunities, although rare and usually short-lived in efficient predictive markets.

Based on the data gathered by \citeauthor{kapp2023improved} in their study on the accuracy of Polymarket, arbitrage opportunities appear to increase as the time frame for the occurrence of the event extends. This is because as the agreed-upon date for the event draws nearer, more information becomes available, and the predicted probability aligns more closely with the actual likelihood, eliminating arbitrage options. Additionally, these data reveal a clear relationship between arbitrage opportunities and the volume of contracts. Markets with a low volume of transactions, lacking a considerable number of agents, tend to be less efficient, and their prices often deviate more from the actual probability of the event, thus creating clear arbitrage opportunities. From the foregoing and the data available on Polymarket, we can assert that in mature markets with high agent participation, arbitrage opportunities tend to be nonexistent.

On the other hand, the presence of a potential arbitrage opportunity between predictive markets is observed, as seen in Image 1 obtained from \citeauthor{lPolack}, where the occurrence of a discovery in the superconductor sector is questioned. Although the question and criteria are the same, there is a discrepancy in the market price between the Metaculus, Manifold, and Polymarket platforms.

    \begin{center}
    \adjustimage{max size={0.9\linewidth}{0.9\paperheight}}{MarketComparison.png}
    \end{center}

This type of divergence is also common in cryptocurrencies predictive markets, because as the user bases and skill levels vary, so do market expectations and pricing. Furthermore, the efficiency of price adjustments is influenced by platform-specific variations in liquidity. Price disparities can also be caused by the dissemination and processing of information, transaction costs, and platform usability, as these factors can influence trader participation and market efficiency. These elements may contribute to the persistence of price disparities, particularly in more closed markets.
