\section{Revising the Assumptions About Predictive Markets}
\label{sec:revising_assumptions}

\subsection{Low Susceptibility to Manipulation}
\label{subsec:low_susceptibility_to_manipulation}

As defined by \cite{buckley2017effect}:
\begin{quote}
    \textit{A manipulation is an attempt by an individual or group of traders to affect the price of contracts being traded on a prediction market in a manner which contradicts their privately held information.}
\end{quote}
    
A prediction market can be manipulated when buyers of shares artificially inflate the demand for a false security regarding the fulfillment of a contract \parencite{choo2022manipulation}. If the results of such a predictive market are used to make policy decisions (exogenous manipulation), for example, the manipulating agents are willing to pay the cost that the deviation from the market prediction implies.
    
According to the literature on predictive markets, it is assumed that these are robust against manipulations. It is based on the argument that, if some agent tries to set a price that is not based on available information, another agent could take advantage of this deviation to make profits \parencite{buckley2017effect}.
    
According to \cite{HANSON2006449}, predictive markets can be subjected to experimentation, as a scenario with a defined set of information and incentives can be reproducible. When subjected to experimentation, studies confirm this hypothesis; however, one of the limitations in these experimental exercises is that participants may incur losses associated with manipulation exercises. This might be true in most predictive markets, but if the manipulating agents value the outcome of the manipulation more than the cost it implies, the hypothesis might not hold \parencite{deck2013affecting}.

In the study conducted by \citeauthor{HANSON2006449}, those agents who were incentivized to manipulate, set prices that were higher compared to those without incentives. However, in their experiment, this manipulation did not have a significant effect on equilibrium prices. Nevertheless, their results depends on the suspicion that the market is being manipulated, meaning that non-manipulating agents are aware of the possible attempt to manipulate and, therefore, correct their prices expectations.
    
On the other hand, \citeauthor{deck2013affecting} designs an experiment with two fundamental differences from that proposed by \Citeauthor{HANSON2006449}: \begin{enumerate*}[label=(\roman*)]
    \item the number of shares is not fixed, but any amount of shares can be bought or sold, and
    \item the manipulating agents have perfect information, and their profits do not depend on the number of shares they hold, but on the market outcome itself.
\end{enumerate*} 
The second point makes manipulators have the maximum incentive to intervene in the market. The conclusion of this study is that the presence of such manipulating agents makes the market unable to aggregate an accurate prediction. Additionally, in this study, it was determined that the manipulation strategy consists of increasing the volume of traded shares compared to the non-manipulated market, achieving this by emptying the Ask offers in the Order Book and, consequently, altering the equilibrium price.